\documentclass[12pt]{report}
\usepackage{graphicx}
\graphicspath{ {images/} }

\title{
{Implementation of publihser / subscriber inter process 
communication model at kernel level}
\newline

{\large Harcourt Butler Technical University, Kanpur}
\newline

{\includegraphics{hbtu_logo.png}}
}
\author{
	Ayush Kumar Shakya
	\and
	Arunima Verma
	\and
	Ashutosh Singh
	\and
	Hemant Singh
}
\date{\today}

\begin{document}
	\maketitle 
	
	\chapter*{Certificate}
	
	This is to certify that 
	\textbf{Ayush Kumar Shakya} (Roll no. 190108017),
	\textbf{Arunima Verma} (Roll no. 190108014),
	\textbf{Ashutosh Singh} (Roll no. 190108015),
	\textbf{Hemant Singh} (Roll no. 190108024),
	students of INFORMATION TECHNOLOGY, Harcourt Butler Technical University, 
	Kanpur are working on their project under my guidance.
	They have a desire for learning and I wish success in their future project work.
	
	\vfill
	\textit{Project guided by}
	\newline
	\textbf{Dr. Prabhat Verma}
	
	\chapter*{Abstract}
	
	Inter process communication (IPC) protocols must be provided by multitasking systems 
	in order to allow communication between processes with varying levels of privilege. 
	
	These mechanisms may be implemented in user space or at the application level, 
	however doing so at the application level has performance concerns because 
	it must make numerous syscalls, which add a lot of overhead. IPC at the kernel level 
	can decrease overheads and improve security. 
	
	Additionally, current kernel-based IPC models like Message Queues 
	require coordination between processes in order to communicate, which might have 
	overheads \cite{citation01} and prevent anonymous communication. 
	
	We believe that each of those problems can be resolved by a kernel level 
	publisher subscriber model.

	\addcontentsline{toc}{chapter}{\protect\numberline{}Certificate}
	\addcontentsline{toc}{chapter}{\protect\numberline{}Abstract}
	\addcontentsline{toc}{chapter}{\protect\numberline{}Bibliography}
	\tableofcontents
	

	\chapter{Introduction}
	In a modern system we have hundreds of processes running at any time and they mostly 
	communicate using application level IPC mechanisms.
	
	The majority of the time, processes choose not to use any communication techniques 
	and often build everything into a single programme, inflating the programmes 
	into large monoliths that are challenging to debug.
	
	Software can be made more reliable and scalable, perhaps across several devices, 
	if it is created as small microservices that utilise effective IPC mechanisms.
	
	\section{Why publisher-subscriber pattern ?}
	
	TODO:
	
	\section{IPC Model}
	\begin{enumerate}
		\item Publishing
		\item Subscription
		\item Notification
	\end{enumerate}
	
	\section{Publishing}
	Any process may send any number of messages on a particular \textbf{topic} or 
	multiple topics, topics may be represented by an inteager.

	some of the topics may be reserved for system use so programs listening to those
	topics can be sure that no malicious program can messages on reserved topics.
	
	This may be implemented as a syscall at kernel level.
				 	
	\section{Subscription}
	Processes can subscribe to one or more than one topics and register a callback
	function to be called when they receive a notification.
	kernel may store these processes in a hash table where \textit{key} is topic 
	and \textit{value} is a linked list or process identifiers or PIDs.
	This registration maybe implemnted as a syscall.c
		
	\section{Notification}
	When a notification arrives on a particular topic, it is added to a queue,
	and kernel dequeues topics one at a time and sends a copy of message to each 
	process \cite{citation04}.
	
	We opted for copying for messages rather than shared memory because
	modern CPUs are quite good at copying small data and having a copy rather than
	shared resourse will help prevent any concurrency related issues such as deadlocks.	

	\section{Cleanup}
	When is process is no longer running it may be removed from the hash table
	so if the same process identifer is assigned to some different process it 
	doesn't receive messages it did not subscribe to.
    
	\chapter{Related Work}
	\section{Apache Kafka}
	Apacha Kafka is distributed event broker that provides publisher-subscriber 
	IPC mechanism at large distributed scale.It more or less achives the functionality 
	that we want to achive.
	
	It is built using Java and Scala and is JVM based therefore quite resourse intensive
	and therefore not suitable for performance critical workloads.
	
	\section{D-Bus}
	D-Bus or Desktop bus is a system bus that provides communication between processes
	using a single system bus it aims to provide standarization communication for 
	linux services.
	
	\subsection{Operating modes} 
	\begin{enumerate}
		\item Request response
		\item Publish subscribe
	\end{enumerate} 
	
	It implemented in user space and therefore cannot be used by kernel specially during
	booting when it needs to stacrt multiple services and a dbus like functionality would
	simplify and speed up the booting process.
	
	There proposals to add dbus like functionality at linux kernel level but nothing
	concrete is merged yet\cite{citation03}.	
	
    \chapter{Progress Report}
	
	\section{Implementation}
	
	\subsection{Kernel}
	To reduce project scope and resource availability, 
	we have developed a 64 bit x86 kernel written in C and will concentrate 
	on optimising our ipc mechanisms for x86 platform only.
	
	The kernel we developed includes a bootloader that loads the kernel and initializes virtual memory
	as well switches to long mode (64 bit mode).
	
	Additionally, the kernel contains several utility functions for both the kernel and userspace programmes, 
	as well as common data structures for heap allocation. 
	
	In addition, the kernel configures the hardware timer, keyboard, and other configurations for CPU interrupts 
	from hardware and software.
	
	\subsection{Process Managment and Scheduler}
	We plan to introduce a process managment and round robin scheduler
	to our kernel.
	
	In order to avoid linking-related activities, we have chosen to employ dummy programmes that 
	our kernel would interpret.
	
	\begin{verbatim}
		Text enclosed inside \texttt{verbatim} environment 
		is printed directly 
		and all \LaTeX{} commands are ignored.
	\end{verbatim}

	
	
	\begin{thebibliography}{9}
	
	\bibitem{citation01}
		S. L. Mirtaheri, E. M. Khaneghah and M. Sharifi, 
		"A Case for Kernel Level Implementation of Inter Process Communication Mechanisms," 
		2008 3rd International Conference on Information and Communication Technologies: From Theory to Applications, 
		2008, pp. 1-7, doi: 10.1109/ICTTA.2008.4530360.
	
	\bibitem{citation02}
		R. Rajkumar, M. Gagliardi and Lui Sha, 
		"The real-time publisher/subscriber inter-process communication model for distributed real-time systems: design and implementation," 
		Proceedings Real-Time Technology and Applications Symposium, 
		1995, pp. 66-75, doi: 10.1109/RTTAS.1995.516203. 

	\bibitem{citation03}
		David Herrmann
		"Bus IPC", https://lists.linuxfoundation.org/pipermail/ksummit-discuss/2016-July/003047.html
		2016
		
	\bibitem{citation04}
		Jochen Liedtke. 1993. 
		Improving IPC by kernel design. In Proceedings of the fourteenth ACM symposium on Operating systems principles (SOSP '93). 
		Association for Computing Machinery, 
		New York, NY, USA, 175–188. 
		https://doi.org/10.1145/168619.168633

	\end{thebibliography}

    
\end{document}